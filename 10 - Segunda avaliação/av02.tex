\documentclass[11pt]{article}
\usepackage[left=1.5cm,right=1.5cm,top=2cm,bottom=2cm]{geometry} % Reduzindo as margens

\usepackage{amsmath}
\usepackage{tikz}
\usepackage{enumitem}

\usepackage[brazilian]{babel}
\usepackage{datetime2}

\usepackage{listings} % Pacote para exibir código fonte
\usepackage{booktabs}
\usepackage{multicol}
\usepackage{fontawesome}
\lstset{
    basicstyle=\small\ttfaky, % Estilo básico do código fonte
    breaklines=true, % Quebra de linha automática
    frame=single, % Borda ao redor do código
    numbers=left % Números de linha à esquerda
}

\usepackage{fancyhdr}
\pagestyle{fancy}
\fancyhf{}

\pgfmathsetseed{\number\pdfrandomseed}

\newcommand{\randominteger}[2]{\pgfmathrandominteger{\random}{#1}{#2}\random}

\newcommand{\sequencia}[2]{
\foreach \i in {1,...,#1} {
        \randominteger{10}{#2}
    }
}
% Cabeçalho
\newcommand{\professor}{Prof. Kennedy Lopes}
\newcommand{\turma}{Algoritmos e Estrutura de Dados II}
\newcommand{\ano}{2023}
\newcommand{\periodo}{1º Semestre}

% Configuração do cabeçalho
\fancyhead{}
\fancyhead[L]{\professor}
\fancyhead[C]{\turma}
\fancyhead[R]{2a Avaliação}

\setlength{\headheight}{14.49998pt}
\addtolength{\topmargin}{-2.49998pt}

% \pagestyle{empty}

\begin{document}

\large \textbf{Nome: \tiny ......................................................................................................................................................................................................}
\begin{enumerate}[label=\textbf{Q\arabic*}]
    \item Foram calculados a frequência com que determinadas palavras aparecem em um deteminado texto:
          \begin{center}
              \begin{tabular}{c|cccccccccc}
                  \hline
                  Letra      & A & B & C & D & E & I & L & M & N & O \\
                  Frequência & 3 & 3 & 1 & 2 & 2 & 5 & 2 & 2 & 4 & 5 \\
                  \hline
              \end{tabular}
          \end{center}
          (a) Construa a árvore de Huffman sobre essa codificação e (b) gere o código para a palavra:

          $$MELANCIA$$

    \item Imagine uma árvore AVL inicialmente vazia na qual desejamos inserir os elementos na ordem especificada:\label{QAVL_1}

          $$X = \{93, 31, 11, 40, 52, 95\}$$

          Após a inserção e balanceamento de cada valor, item a item, indique: (a) a árvore AVL final e (b) a quantidade que cada rotação ocorreu:

          \begin{center}
              \begin{tabular}{|c|c|c|c|c|}
                  \hline
                  Letra      & LL & RL & RR & LR \\
                  \hline
                  Frequência & \  & \  & \  & \  \\
                  \hline
              \end{tabular}
          \end{center}

    \item Realize o processo de inserção dos mesmos valores da questão anterior em uma árvore 23.

    \item Considere uma HEAP como a mostrada logo abaixo e julgue as afirmações em verdadeiras ou falsas:

          $$HEAP = \{97, 88, 84, 72, 55, 44, 37, 30, 26, 12, 18, 20, 25, 14, 8, 10, 6, 15, 5, 9\}$$
          \begin{enumerate}
              \item Esta é uma HEAP-MAX.
              \item Para converter em uma HEAP-MIN basta inverter os valores do jeito que estão posicionados.
              \item Se a prioridade do item 72 mudar para 85, o elemento 84 não mudará de posição.
              \item 88 e 84 estão no mesmo nível, por isso os 4 próximos números tem que ser menores que 84.
              \item A modificação de qualquer um dos elementos 20,25,14,8 para o valor de 86 reduz a prioridade do elemento 84.
          \end{enumerate}

    \item Calcule mapeamentos dos valores:
          $$\{33,14,32,12\}$$
          \begin{enumerate}
              \item Em uma tabela com 4 posições, utilizando o método da dobra no formato binário;
              \item Em uma tabela com 8 posições utilizando método da divisão;
          \end{enumerate}
          Não realize o tratamento de colisões, mas indique na tabela abaixo quantas colisões ocorreram:
          \begin{center}
              \begin{tabular}{|c|c|c|}
                  \hline
                  Alternativa & a) & b) \\
                  \hline
                  Colisões    & \  & \  \\
                  \hline
              \end{tabular}
          \end{center}
\end{enumerate}

\end{document}