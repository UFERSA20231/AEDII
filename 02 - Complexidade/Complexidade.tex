\documentclass[12pt]{beamer}
\setbeamercovered{transparent}
% \beamerdefaultoverlayspecification{<+(1)-| alert@+(1)>}
% Configurações do tema
\usetheme{CambridgeUS}

% Pacotes adicionais
\usepackage{graphicx}

% Informações da apresentação
\title{Análise de Complexidade}
\author{Prof. Kennedy Reurison Lopes}
\date{\today}

% Início do documento
\begin{document}

% Slide de título
\frame{\titlepage}

% Slide de introdução
\begin{frame}
    \frametitle{Introdução}

    \begin{itemize}
        \item Bem-vindos à apresentação sobre a complexidade de realizar uma tarefa!
        \item Hoje discutiremos exemplos de tarefas que podem ser complexas, mesmo não necessariamente estejam diretamente relacionados a algoritmos.
    \end{itemize}
\end{frame}

% Slide 1
\begin{frame}
    \frametitle{Ordenar uma pilha de livros}
    \begin{columns}[T] % Alinhar as colunas no topo
        \begin{column}{0.5\textwidth}
            \begin{itemize}
                \pause \item Imagine uma pilha desorganizada de livros.
                \pause \item A tarefa é organizá-los em ordem alfabética.
                \pause \item A complexidade aumenta à medida que a pilha de livros fica maior.
            \end{itemize}
        \end{column}
        \begin{column}{0.4\textwidth}
            \onslide<1->\begin{figure}[htb]
                \centering
                \includegraphics[width=0.2\textwidth]{livros.jpg}
                \label{fig:livros}
            \end{figure}
        \end{column}
    \end{columns}
\end{frame}

% Slide 2
\begin{frame}
    % \frametitle{Ache o Wally}
    \begin{figure}[htb]
        \centering
        \includegraphics[width=0.9\textwidth]{wally.jpg}
        \label{fig:wally2}
    \end{figure}
\end{frame}

\begin{frame}
    \frametitle{Encontrar um item específico}
    \begin{itemize}\large
        \item Suponha que você precise encontrar um objeto específico em uma sala cheia de itens.
        \item Quanto mais desorganizada a sala e mais objetos houver, mais complexa será a tarefa de localizar o item desejado.
    \end{itemize}
\end{frame}

% Slide 3
\begin{frame}
    \frametitle{Classificar uma coleção de fotos}

    \begin{itemize}
        \item Considere uma grande coleção de fotos digitais a ser classificada em categorias específicas.
        \item Quanto maior a coleção e mais complexas as categorias, mais complexa se torna a tarefa de análise e atribuição de tags apropriadas.
    \end{itemize}

    \begin{figure}[htb]
        \centering
        \includegraphics[width=0.5\textwidth]{fotos.jpg}
        \caption{Exemplo de coleção de fotos a ser classificada.}
        \label{fig:fotos}
    \end{figure}
\end{frame}

% Slide 4
\begin{frame}
    \frametitle{Resolver um quebra-cabeça complexo}

    \begin{itemize}
        \item Pegue um quebra-cabeça desafiador, como um cubo mágico ou um quebra-cabeça de encaixe complexo.
        \item À medida que o número de peças ou a complexidade do quebra-cabeça aumenta, encontrar a solução se torna mais difícil e requer mais tempo e esforço.
    \end{itemize}

    \begin{figure}[htb]
        \centering
        \includegraphics[width=0.2\textwidth]{livros}
        \caption{Exemplo de quebra-cabeça complexo.}
        \label{fig:quebracabeca}
    \end{figure}
\end{frame}

% Slide 5
\begin{frame}
    \frametitle{Planejar uma viagem com múltiplos destinos}

    \begin{itemize}
        \item Ao planejar uma viagem com vários destinos e restrições, como orçamento, tempo, logística, preferências pessoais, entre outros fatores, a complexidade aumenta.
        \item Quanto mais destinos e restrições envolvidos, mais complexo se torna o planejamento.
    \end{itemize}

    \begin{figure}[htb]
        \centering
        \includegraphics[width=0.2\textwidth]{livros}
        \caption{Exemplo de planejamento de viagem com múltiplos destinos.}
        \label{fig:viagem}
    \end{figure}
\end{frame}

% Slide de conclusão
\begin{frame}
    \frametitle{Conclusão}

    \begin{itemize}
        \item Nesta apresentação, exploramos exemplos de tarefas não relacionadas a algoritmos que demonstram a complexidade de realizar uma determinada atividade.
        \item Mesmo em contextos distintos, a complexidade pode aumentar à medida que a quantidade de elementos, restrições ou a dificuldade do problema crescem.
        \item A compreensão dessas complexidades nos ajuda a lidar melhor com tarefas desafiadoras e a buscar soluções eficientes.
    \end{itemize}

    \vspace{1cm}
    Obrigado!

\end{frame}


\end{document}
