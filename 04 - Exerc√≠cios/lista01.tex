\documentclass[12pt]{article}
\usepackage[left=1.5cm,right=1.5cm,top=2cm,bottom=2cm]{geometry} % Reduzindo as margens

\usepackage{amsmath}
\usepackage{tikz}
\usepackage{enumitem}

\usepackage[brazilian]{babel}
\usepackage{datetime2}

\usepackage{listings} % Pacote para exibir código fonte

\lstset{
    basicstyle=\small\ttfamily, % Estilo básico do código fonte
    breaklines=true, % Quebra de linha automática
    frame=single, % Borda ao redor do código
    numbers=left % Números de linha à esquerda
}

\usepackage{fancyhdr}
\pagestyle{fancy}

\pgfmathsetseed{\number\pdfrandomseed}

\newcommand{\randominteger}[2]{\pgfmathrandominteger{\random}{#1}{#2}\random}

\newcommand{\sequencia}[2]{
\foreach \i in {1,...,#1} {
        \randominteger{10}{#2}
    }
}
% Cabeçalho
\newcommand{\professor}{Prof. Kennedy Lopes}
\newcommand{\turma}{Algoritmos e Estrutura de Dados II}
\newcommand{\ano}{2023}
\newcommand{\periodo}{1º Semestre}

% Configuração do cabeçalho
\fancyhead{}
\fancyhead[L]{\professor}
\fancyhead[C]{\turma}
\fancyhead[R]{\ano, \periodo}

\setlength{\headheight}{14.49998pt}
\addtolength{\topmargin}{-2.49998pt}

\begin{document}

\begin{center}
    \LARGE \textbf{Lista de exercícios}
\end{center}
\section*{Recursão}
\begin{enumerate}[label=\textbf{Q\arabic*}]
    \item Através do algoritmo abaixo, calcule o fibonacci de 8. Neste procedimento, verifique quantas vezes o fibonacci(4) foi calculado.
          \begin{lstlisting}[language=C++]
    int fibonacci(int n) {
        if (n <= 1) {
            return n;
        }
        else {
            return fibonacci(n - 1) + fibonacci(n - 2);
        }
    }
        \end{lstlisting}\label{lb:q1}

    \item Generalize a operação da questão~\ref{lb:q1} anterior considerando que estou calculando o fibonacci de qualquer valor maior do que 4.\label{lb:q2}

    \item Generalize a questão~\ref{lb:q2} considerando que estou calculando quantas vezes o fibonacci(n) é calculado para encontrar o fibonacci(m), sendo $m>n$.

    \item No algoritmo abaixo, considere os seguintes tempos:
          \begin{itemize}
              \item Chamada recursiva demora $2ns$
              \item Retorno da chamada recursiva demora $1ns$;
              \item Atribuição e soma demora $0.5ns$;
              \item Divisão e multiplicação demora $1.5ns$
          \end{itemize}
          \begin{lstlisting}[language=C++]
    int funcRecursiva(int n) {
        if (n == 0) {
            return 1;
        }
        return funcRecursiva(n-1) + 1/funcRecursiva(n-1);
    }
\end{lstlisting}
          Calcule o tempo total para funcRecursiva(5).\label{lb:q4}

    \item Refaça o procedimento da questão~\ref{lb:q4} considerando uma modificação no algoritmo:\begin{lstlisting}[language=C++]
    int funcRecursiva(int n) {
        if (n == 0) {
            return 1;
        }
        k = funcRecursiva(n-1);
        return k + 1/k;
    }
\end{lstlisting}

    \item Escreva, passo a passo, a \textbf{execução} do algoritmo fatorial em seu formato recursivo. Evidencie as chamadas recursiva e retornos da recursão.

    \item No calendário gregoriano, \textbf{geralmente} um ano X é bissexto se o ano (x-4) também foi. Este pode ser uma etapa para calcular o algoritmo recursivo para um ano bissexto, mas não está correto por completo. Explique o porquê e apresente uma solução.

    \item Um fractal é uma estrutura geométrica complexa que exibe repetição infinita de padrões semelhantes em diferentes escalas. Um exemplo de fractal simples é o triângulo de Sierpinski. Ele começa com um triângulo equilátero grande. Em seguida, cada lado desse triângulo é dividido em três partes iguais e o triângulo central é removido. Esse processo é repetido para os triângulos restantes, criando uma sequência infinita de triângulos menores que se assemelham ao triângulo original. Desenhe um triângulo de Sierpinski tal qual descrito no texto.\label{lb:q8}
    \item Desenhe um triângulo Sierpinski (descrito na questão~\ref{lb:q8}) computacionalmente através do seguinte algoritmo:
          \begin{enumerate}
              \item Marque os pontos: $A = (0,1); B = (-1,-1); C=(1, -1)$.
              \item Escolha um ponto aleatório P que esteja no interior do triângulo formado por pelos pontos A, B e C.
              \item Marque o ponto médio entre P e um ponto escolhido aleatoriamente entre A, B e C.
          \end{enumerate}
    \item Apresente um algoritmo de recursão com e sem cauda executa a seguinte expressão:
          $$p(x, n) = \prod_{k=0}^{n} (x-k)$$
    \item Apresente versões recursivas de cauda para cada uma das expressões abaixo:
          \begin{enumerate}
              \item $f(n) = n!$
              \item $f(n) = 2 f(n-1) + 3f(n-2), f(0)=1, f(1)=2$
              \item $\displaystyle \sum_{k=1}^{M} k$
          \end{enumerate}
    \item Calcule o $\sin(80)$ considerando como caso base o resultado que $\displaystyle\sin(x) = x - \frac{x^3}{6}$ e que:
          \[
              \begin{cases}
                  \sin(x) & = \sin\left(\frac{x}{3}\right)\left(\frac{3-\tan^2\left(\frac{x}{3}\right)}{1+\tan^2\left(\frac{x}{3}\right)}\right) \\
                  \tan(x) & = \frac{\sin(x)}{\cos(x)}                                                                                            \\
                  \cos(x) & = 1 - \sin\left(\frac{x}{2}\right)                                                                                   \\
              \end{cases}
          \]
    \item Implemente uma versão recursiva dos algoritmos abaixo:
          \begin{enumerate}
              \item Somas sucessivas para calcular o produto de dois números.
              \item Divisão inteira entre dois números através de substrações sucessivas.
              \item Verificação se uma palavra é um palíndromo.
              \item Inversão de uma string.
              \item Geração de todos os números da megasena (6 números entre 1 e 60).
          \end{enumerate}
\end{enumerate}
\section*{Complexidade}
\begin{enumerate}[resume,label=\textbf{Q\arabic*}]
    \item Explique as seguintes afirmações e questionamentos:
          \begin{enumerate}
              \item A função f(n) tem complexidade O($n^2$).
              \item O tempo necessário para execução do algoritmo tem complexidade $\Theta(n \log n)$
              \item Qual o algoritmo mais veloz, o que tem complexidade O(n) ou um outro com $\Theta(n^2)$ ?
          \end{enumerate}
    \item Quais as complexidades de tempo dos algoritmos abaixo (\emph{big O}):
          \begin{enumerate}
              \item $A_1$: Ordenação de uma lista sequencial não ordenada;
              \item $A_2$: Busca de um elemento em uma pilha formado por lista encadeada.
              \item $A_3$: Busca de elementos em uma lista linear encadeada ordenada;
              \item $A_4$: Inserção de elemento numa fila formado por lista encadeada;
              \item Remoção de elemento em uma lista sequencial;
          \end{enumerate}
    \item Quais as complexidades de memória dos algoritmos abaixo (\emph{big O}):
          \begin{enumerate}
              \item Inserção de elementos em uma pilha;
              \item Inserção de elementos em uma fila;
              \item Remoção de elementos em uma pilha;
              \item Remoção de elementos em uma fila;
          \end{enumerate}
    \item Prove se é verdadeiro ou falso:
          \begin{enumerate}
              \item $n^2$ é $O(n^3)$;
              \item $n^3$ é $O(n^2)$;
              \item $log_{10}(n^2)$ é $O(\lg(n))$
              \item $n^2\sin^2(n)$ é $O(n^2)$
          \end{enumerate}
    \item O algoritmo A possui complexidade $O(n^5)$ e o algoritmo B possui complexidade $O(1.5^n)$. Ambos realizam a mesma operação. Qual dos dois você utilizaria?
    \item A quantidade de operações de um algoritmo A é de $T_A(n) = 2n^2 + 5$, do algoritmo B é $T_B(n) = 100n$. Até qual tamanho de problema o algoritmo A é mais eficiente do que o B?
    \item Calcule a complexidade do algoritmo abaixo:\begin{lstlisting}[language=C++]
int f(int n) {
    int s = 0;
    for(int i=0; i<n; i++)
        for(int k=n; k<i; k++)
            s = s + i;
}
\end{lstlisting}
    \item Calcule a complexidade do da função main:\begin{lstlisting}[language=C++]
int f(int n) {
    int s = 0;
    for(int i=0; i<n*n; i++)
        s = s + i;
}
int g(int n){
    f(n/2) * f(n/2);
}
int main(){
    int d;
    scanf("%d\n", d);
    g(d);
}
\end{lstlisting}
    \item Prove se é verdadeiro ou falso:
          \begin{enumerate}
              \item $n^2$ é $\Theta(n^3)$;
              \item $n^3$ é $\Theta(n^2)$;
              \item $log_{10}(n^2)$ é $\Theta(\lg(n))$
              \item $n^2\cos^2(n)$ é $\Theta(n^2)$
          \end{enumerate}
    \item Prove se é verdadeiro ou falso:
          \begin{enumerate}
              \item $n^2$ é $\Omega(n^3)$;
              \item $n^3$ é $\Omega(n^2)$;
              \item $log_{10}(n^2)$ é $\Omega(\lg(n))$
              \item $n^2\cos^2(n)$ é $\Omega(n^2)$
          \end{enumerate}
    \item Julgue as afirmações em (V)erdadeiro ou (F)also:
          \begin{enumerate}
              \item $c_1O(f(n)) = O(c_1*f(n))$
              \item $O(f(n) + g(n)) = O(f(n)*g(n))$
              \item $O(f(n)) + O(g(n)) = O(max(f(n),g(n))))$
              \item $f(n)O(g(n)) = O(g(n))$%falso
          \end{enumerate}
\end{enumerate}
\section*{Conceitos iniciais de Árvores}
\begin{enumerate}[resume, label=\textbf{Q\arabic*}]
    \item Construa uma árvore binária qualquer com os elementos:

          \sequencia{15}{99}\label{lb:q25}

    \item Construa uma árvore estritamente binária com os elementos:

          \sequencia{15}{99}

    \item Construa uma árvore binária cheia com os elementos:

          \sequencia{15}{99}

    \item Construa uma árvore binária completa com os elementos:

          \sequencia{15}{99}

    \item Construa uma árvore binária de busca com os elementos:

          \sequencia{15}{99}

    \item Construa uma árvore binária de busca Zigue-Zague com os elementos:

          \sequencia{15}{99}\label{lb:q30}

\end{enumerate}

\noindent Utilizem das questões ~\ref{lb:q25} a ~\ref{lb:q30} para responder as questões ~\ref{lb:q31} a ~\ref{lb:q36}:

\begin{enumerate}[resume, label=\textbf{Q\arabic*}]

    \item Calcule os sucessores e antecessores dos nós raízes.\label{lb:q31}

    \item Localize o nível de todos os nós.
    
    \item Localize a altura de todos os nós.

    \item Realize o percurso em pré-ordem.

    \item Realize o percurso em pos-ordem.

    \item Realize o percurso em in-ordem.\label{lb:q36}

    \item Julgue (V)erdadeiro ou (F)also:
    \begin{enumerate}
        \item Toda árvore binária cheia é completa.
        \item Toda árvore binária de busca Zigue-Zague possui como raiz o menor elemento.
        \item É possível uma árvore ser simultaneamente: Estitamente binária, Cheia, Completa e Zigue-Zague.
        \item É possível uma árvore ser simultaneamente: Estitamente binária, Cheia, Completa. Mas ser também Zigue-Zague é impossível.
        \item Se o antecessor e sucessor de um nó são irmãos, então o nó é pai dos dois.
        \item O tio de um nó é ancestral a ele.
        \item O neto de um nó é um descendente dele.
    \end{enumerate}
    
    \item Qual a altura de uma árvore cheia que possui N=\randominteger{100}{1000} nós?
    
    \item Quantos nós faltam para uma árvore cheia com N=\randominteger{100}{1000} nós se torne uma árvore completa?
    
    \item Quantos nós precisariam ser removidos para que uma árvore cheia com N=\randominteger{100}{1000} nós se torne uma árvore completa?
    
    \item Qual a maior altura do nó raiz em uma árvore estritamente binária com N=\randominteger{100}{1000} nós.

\end{enumerate}

\end{document}
