\documentclass[12pt]{beamer}
% \setbeamercovered{transparent}
\usepackage{tikzpeople}
\usepackage[portuguese]{babel}

\beamerdefaultoverlayspecification{<+(1)-| alert@+(1)>}
\usetheme{Boadilla}
\usecolortheme{seahorse}
\usefonttheme{structurebold}
\usepackage{xcolor}

\title{Árvore Balanceadas}
\subtitle{AVL}
\author{Prof. Kennedy Reurison Lopes}
\date{\today}

\usepackage{forest}

\begin{document}

\frame{\titlepage}

\newcommand{\floresta}[1]{
    \begin{center}
        \begin{forest}
            for tree={circle, minimum size=0.5em,edge={->},l sep+=0em,s sep+=2em,font=\small}
            #1
        \end{forest}
    \end{center}
}
\begin{frame}[t]
    \frametitle{Introdução}

    Vimos que a ordem de inserção interfere o formato de uma árvore binária:
    \vfill
    \begin{columns}[T]
        \begin{column}{0.45\textwidth}
            Inserir: 1-2-3-4\vfill
            % \floresta{[1[2[3[4]]]]}
            \floresta{[1[,edge=white]
                            [2
                                    [,edge=white]
                                    [3
                                            [,edge=white]
                                            [4]]]]}
        \end{column}\hfill
        \begin{column}{0.45\textwidth}
            Inserir: 2-1-4-3\vfill
            \floresta{[2[1][4[3][,edge=white]]]}
        \end{column}
    \end{columns}

\end{frame}

\hyphenation{es-tri-ta-men-te}
\begin{frame}[t]
    \frametitle{Introdução}

    Em casos extremos a árvore tem comportamento de uma estritamente igual a uma sequencial ou binária.

    Portanto:
    \begin{itemize}
        \item O balanceamento é importante;
        \item Caso possível, podemos escolher a ordem de inserção;
        \item Caso não seja possível, precisamos re-estruturar a árvore a medida que os elementos forem inseridos.
    \end{itemize}

\end{frame}


\begin{frame}[t]
    \frametitle{Árvores Balanceadas}
    \begin{center}
        \begin{tikzpicture}[meuestilo/.style={ellipse callout, very thin, draw, text width=2.5cm, font=\small},
                meuestiloA/.style={meuestilo,draw=white, fill=red!10, xshift=4cm,yshift=4cm,callout absolute pointer={(a.mouth)}},
                meuestiloB/.style={meuestilo,draw=white, fill=green!10, xshift=5cm,yshift=4cm,, callout absolute pointer={(b.mouth)}}]
            \draw[white] (0,0) rectangle (10,6);
            \node<1->[name=a,anchor=south west,saturated, shape=businessman,monitor,shirt=purple, minimum size=2cm] at (0,0){};
            \node<2>[meuestiloA,callout absolute pointer={(a.mouth)}] {O que é isso?};
            \node<2->[name=b,anchor=south east,shape=graduate,minimum size=2cm,mirrored] at (10,0){};
            \node<3>[meuestiloB] {Calma que eu explico.};
            \node<4>[meuestiloB] {Basta que as árvores sejam \textbf{completas}!};
            \node<5>[meuestiloA] {Então só posso usar árvores com $2^n - 1$ nós?};
            \node<6>[meuestiloB] {Sendo \textbf{cheia} já é \emph{suficiente}};
            \node<7>[meuestiloA] {Ainda estou achando muito difícil};
            \node<8>[meuestiloA] {A ordem de inserção ainda determina se a árvore será \textbf{cheia} ou \textbf{completa}};
            \node<9>[meuestiloB,text width=3cm] {Então vamos definir o conceito de \textbf{Fator de Balanceameto}};
            \node<10>[meuestiloB,text width=3cm] {A partir deste, apresento como balancear a árvore};
            \node<11>[meuestiloB,text width=3.5cm] {Não serão mais exigências: Árvores Completas};
            \node<11>[meuestiloB,text width=3.5cm] {Cheias};
            \node<11>[meuestiloB,text width=3.5cm] {Ou Ordem de Inserção};
        \end{tikzpicture}
    \end{center}

\end{frame}

\begin{frame}[t]
    \frametitle{Fator de balanceamento}

    \begin{block}{Fator de Balanceamento (FB)}
        Considere $h_{esq}(n)$ e $h_{dir}(n)$ de um nó $n$, então o $FB(n)$ é:
        $$FB(n) = h_{esq} - h_{dir}$$
    \end{block}

    \pause Corolário 1:

    Uma árvore Cheia possui o FB igual a zero para todos os nós.

    \vfill
    \pause Corolário 2:
    Toda árvore Cheia é balanceada, mas nem toda árvore balanceada é cheia.



\end{frame}
\begin{frame}
    \frametitle{Teste}
    \begin{center}
        \begin{forest}
            for tree={circle,draw=white,fill=blue!10, minimum size=0.5em,edge={->},l sep+=0.1em,s sep+=0.3em},
            [a
                        [b
                                [d
                                        [h]
                                        [i]
                                ]
                                [e
                                        [j]
                                        [k]
                                ]
                        ]
                        [c
                                [f[l][,edge=blue!0,fill=white]]
                                [g
                                        [n]
                                        [o]
                                ]
                        ]
                ]
        \end{forest}
    \end{center}
\end{frame}


\end{document}
