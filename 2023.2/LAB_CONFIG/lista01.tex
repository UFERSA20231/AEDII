\documentclass[12pt]{article}
\usepackage[left=1.5cm,right=1.5cm,top=3cm,bottom=2cm]{geometry} % Reduzindo as margens

\usepackage{amsmath}
\usepackage{tikz}
\usepackage{enumitem}
\usepackage{hyperref}
\usepackage{xcolor}

\usepackage[brazilian]{babel}
\usepackage{datetime2}

\usepackage{listings} 

\lstdefinestyle{codigoC}{
  language=C,
  basicstyle=\ttfamily\small,
  numbers=left,
  numberstyle=\tiny,
  numbersep=5pt,
  keywordstyle=\color{blue},
  morekeywords={int, char, float, double, if, else, for, while, return},
  stringstyle=\color{red!80!black!80},
  showstringspaces=false,
  breaklines=true,
  frame=single
}

\lstset{language=C++,
                basicstyle=\ttfamily,
                keywordstyle=\color{blue}\ttfamily,
                stringstyle=\color{red}\ttfamily,
                commentstyle=\color{green!60!black},
                morecomment=[l][\color{black!50}]{\#}
}

\usepackage{fancyhdr}
\pagestyle{fancy}

\pgfmathsetseed{\number\pdfrandomseed}

\newcommand{\randominteger}[2]{\pgfmathrandominteger{\random}{#1}{#2}\random}

\newcommand{\sequencia}[2]{
\foreach \i in {1,...,#1} {
        \randominteger{10}{#2}
    }
}
% Cabeçalho
\newcommand{\professor}{Huliane Medeiros\\Kennedy Lopes\\Marcos Mikael}
\newcommand{\turma}{Laboratório de Algoritmos\\ e Estrutura de Dados II}
\newcommand{\ano}{BTI/BICT\\}
\newcommand{\periodo}{2023.2}

% Configuração do cabeçalho
\fancyhead{}
\fancyhead[L]{\professor}
\fancyhead[C]{\turma}
\fancyhead[R]{\ano \periodo}

\setlength{\headheight}{43.49998pt}
\addtolength{\topmargin}{-2.49998pt}

\begin{document}
\begin{center}
    \LARGE \textbf{Lista de exercícios (CONFIG)}
\end{center}

\subsection*{Preparação do ambiente}
\begin{itemize}
    \item Configure uma máquina virtual com o sistema linux:
          \begin{itemize}
              \item Opção 1: Utilize uma máquina com o linux instalado;
              \item Opção 2: Instale uma máquina virtual para instalar o linux: \href{https://www.virtualbox.org/}{https://www.virtualbox.org/}
              \item Opção 3: Instale uma versão nativa do linux (recomendado);
          \end{itemize}
\end{itemize}

\subsection*{Instalações}
\begin{itemize}
    \item Instale o vscode:
          \begin{verbatim}
        sudo snap install --classic code
    \end{verbatim}
    \item instale o GCC, G++ e o GDB:
          \begin{verbatim}
        sudo apt-get install build-essential gdb
    \end{verbatim}
    \item Instale o GIT:
          \begin{verbatim}
        sudo apt-get install git
    \end{verbatim}
    \item Verifique se está tudo corretamente instalado:
          \begin{verbatim}
        gcc --version
        g++ --version
        gdb --version
        code --version
        git --version
    \end{verbatim}
\end{itemize}
\textbf{Atenção:}  Se algum erro ocorrer, notifique os professores ou o monitor da disciplina.
\subsection*{Testando o ambiente}
\begin{itemize}
    \item Excute o vscode;
    \item Crie uma função simples (helloWorld.c);
    \item Execute o código com:
    \begin{verbatim}
        gcc -o main helloWorld.c -g
    \end{verbatim}
    \item Execute o código com:
    \begin{verbatim}
        ./main
    \end{verbatim}
    Rode em depuração com:
    \begin{verbatim}
        gdb main
    \end{verbatim}
    \textbf{Atenção:}  Se algum erro ocorrer, notifique os professores ou o monitor da disciplina.
\end{itemize}

\end{document}