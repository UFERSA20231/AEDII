\documentclass[12pt]{article}
\usepackage[left=1.5cm,right=1.5cm,top=2cm,bottom=2cm]{geometry} % Reduzindo as margens

\usepackage{enumitem}

\usepackage{listings} % Pacote para exibir código fonte

\lstset{
    basicstyle=\small\ttfamily, % Estilo básico do código fonte
    breaklines=true, % Quebra de linha automática
    frame=single, % Borda ao redor do código
    numbers=left % Números de linha à esquerda
}

\usepackage{fancyhdr}
\pagestyle{fancy}

% Cabeçalho
\newcommand{\professor}{Prof. Kennedy Lopes}
\newcommand{\turma}{Algoritmos e Estrutura de Dados II}
\newcommand{\ano}{2023}
\newcommand{\periodo}{1º Semestre}

% Configuração do cabeçalho
\fancyhead{}
\fancyhead[L]{\professor}
\fancyhead[C]{\turma}
\fancyhead[R]{\ano, \periodo}

\setlength{\headheight}{14.49998pt}
\addtolength{\topmargin}{-2.49998pt}

\begin{document}

\begin{center}
    \LARGE \textbf{Lista de exercícios}
\end{center}
\section*{Recursão}
\begin{enumerate}[label=\textbf{Q\arabic*.}]
    \item Através do algoritmo abaixo, calcule o fibonacci de 8. Neste procedimento, verifique quantas vezes o fibonacci(4) foi calculado.
    \input{q1.tex}\label{lb:q1}
    
    \item Generalize a operação da questão~\ref{lb:q1} anterior considerando que estou calculando o fibonacci de qualquer valor maior do que 4.\label{lb:q2}
    
    \item Generalize a questão~\ref{lb:q2} considerando que estou calculando quantas vezes o fibonacci(n) é calculado para encontrar o fibonacci(m), sendo $m>n$.
    
    \item No algoritmo abaixo, considere os seguintes tempos:
    \begin{itemize}
        \item Chamada recursiva demora $2ns$
        \item Retorno da chamada recursiva demora $1ns$;
        \item Atribuição e soma demora $0.5ns$;
        \item Divisão e multiplicação demora $1.5ns$
    \end{itemize}
    \begin{lstlisting}[language=C++]
    int funcRecursiva(int n) {
        if (n == 0) {
            return 1;
        }
        return funcRecursiva(n-1) + 1/funcRecursiva(n-1);
    }
\end{lstlisting}
    Calcule o tempo total para funcRecursiva(5).\label{lb:q4}

    \item Refaça o procedimento da questão~\ref{lb:q4} considerando uma modificação no algoritmo:\begin{lstlisting}[language=C++]
    int funcRecursiva(int n) {
        if (n == 0) {
            return 1;
        }
        k = funcRecursiva(n-1);
        return k + 1/k;
    }
\end{lstlisting}

    \item Escreva, passo a passo, a \textbf{execução} do algoritmo fatorial em seu formato recursivo. Evidencie as chamadas recursiva e retornos da recursão.
    
    \item No calendário gregoriano, \textbf{geralmente} um ano X é bissexto se o ano (x-4) também foi. Este pode ser uma etapa para calcular o algoritmo recursivo para um ano bissexto, mas não está correto por completo. Explique o porquê e apresente uma solução.
    
    \item 

\end{enumerate}
\section*{Complexidade}
\begin{enumerate}[resume,label=\textbf{Q\arabic*.}]
    \item 
    \item 
    \item 
\end{enumerate}
\section*{Conceitos iniciais de Árvores}

\end{document}
