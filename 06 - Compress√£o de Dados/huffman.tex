\documentclass[10pt]{beamer}
% \setbeamercovered{transparent}
\beamerdefaultoverlayspecification{<+(1)-| alert@+(1)>}
\usetheme{Boadilla}
\usecolortheme{seahorse}
\usefonttheme{structurebold}
\usepackage{booktabs}
\usepackage{array}
\usepackage{rotating}
\usepackage{xcolor}
\usepackage{xstring}
\usepackage{dirtree}
\usepackage{enumitem}

\usepackage{tikz}
\usepackage{expl3}


\title{Compressão de Dados}
\subtitle{Algoritmo de Huffman}
\author{Prof. Kennedy Reurison Lopes}
\date{\today}

\begin{document}

\frame{\titlepage}


\definecolor{corA1}{RGB}{0, 0, 0}
\definecolor{corA2}{RGB}{0, 225, 0}
\definecolor{corA3}{RGB}{0, 0, 255}
\definecolor{mixA}{RGB}{0,225,255}
% \definecolor{mixA}{rgb}{0.7,102,25}

\definecolor{corB1}{RGB}{98, 0, 0}
\definecolor{corB2}{RGB}{0, 255, 0}
\definecolor{corB3}{RGB}{0, 0, 0}
\definecolor{mixB}{RGB}{95,255,0}

\definecolor{corC1}{RGB}{255, 0, 0}
\definecolor{corC2}{RGB}{0, 30, 0}
\definecolor{corC3}{RGB}{0, 0, 0}
\definecolor{mixC}{RGB}{255,30,0}

\definecolor{corD1}{rgb}{157, 0, 0}
\definecolor{corD2}{rgb}{0, 0, 0}
\definecolor{corD3}{rgb}{0, 0, 255}
\definecolor{mixD}{RGB}{157,0,255}
\begin{frame}[t]
    \frametitle{Codificação}
    Uma imagem numa tela é composta por uma sequência de pixels no formato RGB.
    \vfill
    \begin{center}
        \def\l{2}
        \def\ml{0.25}
        \def\mdX{6}
        \def\mdY{3.5}
        \def\nl{\l/9}
        \begin{tikzpicture}[pixel/.style={very thin,rounded corners=3pt, color=white}]

            \draw<1-4>[white] (-\l,-\l) rectangle (\mdX+\ml,\mdY+\ml);

            \fill<1-4>[mixA] (\mdX,\mdY) rectangle (\ml +\mdX,\ml+\mdY);
            \fill<2-4>[mixB] (\mdX,\mdY) rectangle (\ml +\mdX,-\ml+\mdY);
            \fill<3-4>[mixC] (\mdX,\mdY) rectangle (-\ml+\mdX,-\ml+\mdY);
            \fill<4-4>[mixD] (\mdX,\mdY) rectangle (-\ml+\mdX, \ml+\mdY);

            \draw<4>[dashed,very thin] (-\l, \l) -- (-\ml +\mdX, \ml+\mdY);
            \draw<4>[dashed,very thin] (\l, -\l) -- (\ml +\mdX, -\ml+\mdY);

            \draw<1->[dotted,color=black] (0,0) rectangle (\l,\l);
            \draw<2->[dotted,color=black] (0,0) rectangle (\l,-\l);
            \draw<3->[dotted,color=black] (0,0) rectangle (-\l,-\l);
            \draw<4->[dotted,color=black] (0,0) rectangle (-\l,\l);

            \fill<1->[pixel,fill=corA1] (1*\nl, \nl) rectangle (2*\nl, 8*\nl);
            \fill<1->[pixel,fill=corA2] (4*\nl, \nl) rectangle (5*\nl, 8*\nl);
            \fill<1->[pixel,fill=corA3] (7*\nl, \nl) rectangle (8*\nl, 8*\nl);

            \fill<2->[pixel,fill=corC1] (1*\nl, -8*\nl) rectangle (2*\nl,-1*\nl);
            \fill<2->[pixel,fill=corC2] (4*\nl, -8*\nl) rectangle (5*\nl,-1*\nl);
            \fill<2->[pixel,fill=corC3] (7*\nl, -8*\nl) rectangle (8*\nl,-1*\nl);

            \fill<3->[pixel,fill=corD1] (-7*\nl, -8*\nl) rectangle (-8*\nl,-1*\nl);
            \fill<3->[pixel,fill=corD2] (-4*\nl, -8*\nl) rectangle (-5*\nl,-1*\nl);
            \fill<3->[pixel,fill=corD3] (-1*\nl, -8*\nl) rectangle (-2*\nl,-1*\nl);

            \fill<4-5>[pixel,fill=corB1] (-7*\nl, \nl) rectangle (-8*\nl, 8*\nl);
            \fill<4-5>[pixel,fill=corB2] (-4*\nl, \nl) rectangle (-5*\nl, 8*\nl);
            \fill<4-5>[pixel,fill=corB3] (-1*\nl, \nl) rectangle (-2*\nl, 8*\nl);
        \end{tikzpicture}
    \end{center}
\end{frame}

\def\nota1{Um arquivo RAW contém todos os dados de uma imagem não compactados e não processados capturados por um scanner ou pelos sensores de uma câmera digital. As fotos feitas no formato RAW contêm alto nível de detalhes, são grandes e não têm perdas de qualidade.}
\begin{frame}[t]
    \frametitle{Codificação - Imagem RAW}
    Considere uma imagem no formato \emph{raw}\footnote{\nota1}. O arquivo desta imagem preserva a qualidade original da foto, sem perder nem um \emph{bit} de informação.\vfill

    Uma imagem em 4K desta imagem tem a seguinte quantidade de pixels:

    \[P =  3840 \times 2160 = 8.294.400\ \text{pixels}\]

    \vfill

    Cada Pixel contem uma cor no formato RGB(3 \emph{bytes}):

    \[S = 3840 \times 2160 \times 3 = 24.883.200\ \text{bytes}\]

    \vfill

    Cada $1.024$ bytes corresponde a 1kB e cada $1.024kB$ corresponde a $1$MB. Portanto, uma imagem RAW em formato 4K contém aproximadamente:

    \[S \simeq 23.73\ \text{MB}\]
\end{frame}

\begin{frame}[t]
    \frametitle{Video 4K}

    Imagine a situação que os dados são transmitidos pixel a pixel no formato 4K. Uma hora deste filme custaria aproximagamente:
    \pause
    $$S = (1h)\times\left(\frac{23.73\ MB}{frame}\right) \times
        \pause\left(\frac{120\ frame}{s}\right)\times
        \pause\left(\frac{60 min}{1h}\right)\times
        \pause\left(\frac{60s}{1min}\right) $$

    \pause
    $$S = 23.73\times 120\times 60\times 60\times\frac{MB\times h\times frame\times min\times s}{frame\times s\times h\times min}$$

    \pause
    $$S = 1.025.136.000\ \text{MB} \simeq 1.001.110\ \text{GB}\simeq 977\ \text{TB}$$

    \pause
    Obviamente os filmes não ocupam este espaço no disco. Então como é possível armazenar um filme em um Blue-Ray\footnote{Mídia digital capaz de armazenar até 25GB de dados}, ou mesmo transferir o filme em 4K no \emph{streaming}?\vfill

    \pause
    \begin{flushbottom}
        \begin{minipage}{\textwidth}
            \ {\raggedleft
            \begin{turn}{180}
                \tiny Resposta: Compactação.
            \end{turn}
            \par}
        \end{minipage}
    \end{flushbottom}
\end{frame}

\begin{frame}[t]
    \frametitle{Codificação - Imagem RAW}
    Algumas considerações:\vfill
    \begin{itemize}
        \setlength\itemsep{2em}
        \item Como é um byte para cada cor, o computador pode gerar até:
              $$(2^8)^3 = 2^{24} = 16.777.216\ \text{cores}$$
        \item O ser humano consegue enxergar \emph{apenas} 1 milhão de cores.
        \item Em uma imagem, não são utilizadas todas estas cores. Exemplos:
              \begin{itemize}
                  \item Uma imagem $800\times 600$ precisa ter no máximo 480.000 cores;
                  \item A resolução HD precisa de no máximo $1280\times 720 = 921.600$ cores.
                  \item A própria resolução 4K precisa apenas de $8.294.400$ cores.
              \end{itemize}
        \item Existem muitos pixels repetidos (cores repetidas).
    \end{itemize}
\end{frame}

\newcommand{\asc}[1]{\number`#1}
% Novo comando para obter o valor ASCII de uma palavra
\newcommand{\asciiValueWord}[1]{%
    \begingroup
    \def\result{}%
    \expandafter\asciiValueWordAux#1\relax
    \result
    \endgroup
}
\def\asciiValueWordAux#1{%
    \ifx#1\relax
    \else
        \ifx\result\empty
            \edef\result{\number`#1}%
        \else
            \edef\result{\result - \number`#1}%
        \fi
        \expandafter\asciiValueWordAux
    \fi
}


\begin{frame}[t]
    \frametitle{Codificação de texto}
    Um texto é composto por uma sequência de caracteres (imprimíveis e não-imprimíveis) sendo, cada caractere representado por uma sequência de bits. Exemplo:

    $$K - E - N - N - E - D - Y$$

    Utilizando a codificação ASCII pode ser:

    $$\asciiValueWord{Kennedy}$$

    Sendo que cada símbolo deste é representado por uma sequência binária com 8 bits.

    Cada caractere possui exatamente 1byte de representação. Desta forma, o meu nome ocupa:

    $$56\ \text{bits}$$

\end{frame}

\newcommand{\separarPorTravessao}[1]{%
    \@tfor\next:=#1\do{%
        \next--\space
    }%
}
\makeatother

\ExplSyntaxOn
\NewDocumentCommand{\separarPorHifen}{m}
{
    \tl_map_inline:nn { #1 } { ##1\ \ }
}
\ExplSyntaxOff

\begin{frame}[t]
    \frametitle{Codificação de texto}
    Vamos considerar um texto maior: \textbf{Um prato de trigo para tres tigres tristes}:

    $$\separarPorHifen{Um\_prato}$$
    $$\asciiValueWord{Um\_prato}$$

    $$\separarPorHifen{de\_trigo}$$
    $$\asciiValueWord{de\_trigo}$$

    $$\separarPorHifen{para\_tres}$$
    $$\asciiValueWord{para\_tres}$$

    $$\separarPorHifen{tigres\_tristes}$$
    $$\asciiValueWord{tigres\_tristes}$$
\end{frame}


\ExplSyntaxOn
\NewDocumentCommand{\contararCaracteres}{m}
{
    \str_count:n { #1 }
}
\ExplSyntaxOff

\ExplSyntaxOn
\NewDocumentCommand{\contararAlgoritmosDistintos}{m}
{
    \seq_new:N \l_char_seq
    \tl_map_inline:nn { #1 }
    {
        \seq_put_right:Nn \l_char_seq { ##1 }
    }
    \seq_remove_duplicates:N \l_char_seq
    \int_compare:nT {\seq_count:N \l_char_seq > 0}
    {
        \seq_count:N \l_char_seq
    }
}
\ExplSyntaxOff


\begin{frame}[t]
    \frametitle{Codificação de texto}
    Vamos considerar um texto maior: \textbf{Um prato de trigo para tres tigres tristes}:
    Este texto contem:
    \begin{itemize}
        \item $42$ caracteres.
        \item $13$ caracteres distintos.
        \item O texto ocupa $336$ bits.
    \end{itemize}

    \pause É possível ocupar menos espaço?

    \pause Vamos analisar a frequência dos caracteres:



\end{frame}

\def\texto{Um-prato-de-trigo-para-tres-tigres-tristes}
\newcommand{\contar}[1]{%
    \StrCount{\texto}{#1}%
}

\begin{frame}[t]
    \frametitle{Análise de Frequência}
    Frequência dos caracteres:

    \pause\begin{table}
        \resizebox{0.8\textwidth}{!}{%
            \begin{tabular}{c|c|c|c|c|c|c|c|c|c|c|c|c}
                U          & m          & -          & p          & r          & a          & t          & o          & d          & e          & i          & g          & s          \\
                \hline
                \contar{U} & \contar{m} & \contar{-} & \contar{p} & \contar{r} & \contar{a} & \contar{t} & \contar{o} & \contar{d} & \contar{e} & \contar{i} & \contar{g} & \contar{s} \\
            \end{tabular}%
        }
    \end{table}

    Ordenando as frequências:

    \pause\begin{table}
        \resizebox{0.8\textwidth}{!}{%
            \begin{tabular}{c|c|c|c|c|c|c|c|c|c|c|c|c}
                U          & m          & d          & p          & o          & g          & a          & i          & e          & s          & r          & t          & -          \\
                \hline
                \contar{U} & \contar{m} & \contar{d} & \contar{p} & \contar{o} & \contar{g} & \contar{a} & \contar{i} & \contar{e} & \contar{s} & \contar{r} & \contar{t} & \contar{-} \\
            \end{tabular}%
        }
    \end{table}

    Agrupando as menores frequências:

    \pause\begin{table}
        \resizebox{0.8\textwidth}{!}{%
            \begin{tabular}{c|c|c|c|c|c|c|c|c|c|c|c}
                U          & $\{d,m\}$ & p          & o          & g          & a          & i          & e          & s          & r          & t          & -          \\
                \hline
                \contar{U} & 2         & \contar{p} & \contar{o} & \contar{g} & \contar{a} & \contar{i} & \contar{e} & \contar{s} & \contar{r} & \contar{t} & \contar{-} \\
            \end{tabular}%
        }
    \end{table}

\end{frame}


\begin{frame}
    \frametitle{Análise de Frequência}
    O algoritmo continua (agrupando os menores) até não for mais possível continuar:
    \vfill
    Agrupando(U e $\{d,m\}$):

    \pause\begin{table}
        \resizebox{0.8\textwidth}{!}{%
            \begin{tabular}{c|c|c|c|c|c|c|c|c|c|c}
                $\{U, d,m\}$ & p          & o          & g          & a          & i          & e          & s          & r          & t          & -          \\
                \hline
                3            & \contar{p} & \contar{o} & \contar{g} & \contar{a} & \contar{i} & \contar{e} & \contar{s} & \contar{r} & \contar{t} & \contar{-} \\
            \end{tabular}%
        }
    \end{table}

    Ordenando:

    \pause\begin{table}
        \resizebox{0.8\textwidth}{!}{%
            \begin{tabular}{c|c|c|c|c|c|c|c|c|c|c}
                g          & p          & o          & $\{U, d,m\}$ & a          & i          & e          & s          & r          & t          & -          \\
                \hline
                \contar{g} & \contar{p} & \contar{o} & 3            & \contar{a} & \contar{i} & \contar{e} & \contar{s} & \contar{r} & \contar{t} & \contar{-} \\
            \end{tabular}%
        }
    \end{table}

\end{frame}


\begin{frame}[t]
    \frametitle{Análise de Frequência}
    \pause\begin{table}
        \resizebox{0.8\textwidth}{!}{%
            \begin{tabular}{c|c|c|c|c|c|c|c|c|c}
                g          & $\{p,o\}$ & $\{U, d,m\}$ & a          & i          & e          & s          & r          & t          & -          \\
                \hline
                \contar{g} & 4         & 3            & \contar{a} & \contar{i} & \contar{e} & \contar{s} & \contar{r} & \contar{t} & \contar{-} \\
            \end{tabular}%
        }
    \end{table}

    \pause\begin{table}
        \resizebox{0.8\textwidth}{!}{%
            \begin{tabular}{c|c|c|c|c|c|c|c|c}
                $\{g,a\}$ & $\{p,o\}$ & $\{U, d,m\}$ & i          & e          & s          & r          & t          & -          \\
                \hline
                5         & 4         & 3            & \contar{i} & \contar{e} & \contar{s} & \contar{r} & \contar{t} & \contar{-} \\
            \end{tabular}%
        }
    \end{table}

    \pause\begin{table}
        \resizebox{0.8\textwidth}{!}{%
            \begin{tabular}{c|c|c|c|c|c|c|c}
                $\{g,a\}$ & $\{p,o\}$ & $\{U,d,m,i\}$ & e          & s          & r          & t          & -          \\
                \hline
                5         & 4         & 6             & \contar{e} & \contar{s} & \contar{r} & \contar{t} & \contar{-} \\
            \end{tabular}%
        }
    \end{table}

    \pause\begin{table}
        \resizebox{0.8\textwidth}{!}{%
            \begin{tabular}{c|c|c|c|c|c|c}
                $\{g,a\}$ & $\{p,o\}$ & $\{U,d,m,i\}$ & $\{e,s\}$ & r          & t          & -          \\
                \hline
                5         & 4         & 6             & 8         & \contar{r} & \contar{t} & \contar{-} \\
            \end{tabular}%
        }
    \end{table}
\end{frame}

\begin{frame}[t]
    \frametitle{Análise de Frequência}
    \pause\begin{table}
        \resizebox{0.8\textwidth}{!}{%
            \begin{tabular}{c|c|c|c|c|c}
                $\{g,a,p,o\}$ & $\{U,d,m,i\}$ & $\{e,s\}$ & r          & t          & -          \\
                \hline
                9             & 6             & 8         & \contar{r} & \contar{t} & \contar{-} \\
            \end{tabular}%
        }
    \end{table}

    \pause\begin{table}
        \resizebox{0.8\textwidth}{!}{%
            \begin{tabular}{c|c|c|c|c}
                $\{g,a,p,o\}$ & $\{U,d,m,i\}$ & $\{e,s\}$ & $\{r,t\}$ & -          \\
                \hline
                9             & 6             & 8         & 12        & \contar{-} \\
            \end{tabular}%
        }
    \end{table}

    \pause\begin{table}
        \resizebox{0.8\textwidth}{!}{%
            \begin{tabular}{c|c|c|c}
                $\{g,a,p,o\}$ & $\{U,d,m,i,-\}$ & $\{e,s\}$ & $\{r,t\}$ \\
                \hline
                9             & 13              & 8         & 12        \\
            \end{tabular}%
        }
    \end{table}

    \pause\begin{table}
        \resizebox{0.8\textwidth}{!}{%
            \begin{tabular}{c|c|c}
                $\{g,a,p,o,e,s\}$ & $\{U,d,m,i,-\}$ & $\{r,t\}$ \\
                \hline
                17                & 13              & 12        \\
            \end{tabular}%
        }
    \end{table}

\end{frame}

\begin{frame}[t]
    \frametitle{Análise de Frequência}
    \pause\begin{table}
        \resizebox{0.7\textwidth}{!}{%
            \begin{tabular}{c|c}
                $\{g,a,p,o,e,s\}$ & $\{U,d,m,i,-,r,t\}$ \\
                \hline
                17                & 25                  \\
            \end{tabular}%
        }
    \end{table}
    \pause\begin{table}
        \resizebox{0.7\textwidth}{!}{%
            \begin{tabular}{c}
                $\{g,a,p,o,e,s,U,d,m,i,-,r,t\}$ \\
                \hline
                42                              \\
            \end{tabular}%
        }
    \end{table}

\end{frame}

\newcommand{\duascolunas}[2]{
    \begin{columns}
        \begin{column}{0.72\textwidth}
            #1
        \end{column}
        \begin{column}{0.28\textwidth}
            #2
        \end{column}
    \end{columns}
}

\begin{frame}
    \frametitle{Construção da árvore}
    \begin{columns}[t]
        \begin{column}[t]{0.30\textwidth}
            \dirtree{%
                .1 gapoesUdmi-r(42).
                .2 gapoes(0,17).
                .3 gapo(0,9).
                .4 ga(0,5).
                .5 g(0,2).
                .5 a(1,3).
                .4 po(1,4).
                .5 p(0,2).
                .5 o(1,2).
                .3 es(1,8).
                .4 e(0,4).
                .4 s(1,4).
            }
        \end{column}
        \begin{column}[t]{0.40\textwidth}
            \dirtree{%
                .1 gapoesUdmi-r(42).
                .2 Udmi-rt(1,25).
                .3 Udmi-(0,13).
                .4 Udmi(0,6).
                .5 Udm(0,3).
                .6 U(0,1).
                .6 dm(1,2).
                .7 d(0,1).
                .7 m(1,1).
                .5 i(1,3).
                .4 -(0,7).
                .3 rt(1,12).
                .4 r(0,6).
                .4 t(1,6).
            }
        \end{column}\hfill
        \begin{column}[t]{0.30\textwidth}
            \begin{itemize}\setlength\itemsep{0.1em}
                \item [U]$\rightarrow 10000$
                \item [m]$\rightarrow 1000011$
                \item [d]$\rightarrow 1000010$
                \item [p]$\rightarrow 0010$
                \item [o]$\rightarrow 0011$
                \item [g]$\rightarrow 0000$
                \item [a]$\rightarrow 0001$
                \item [i]$\rightarrow 1001$
                \item [e]$\rightarrow 010$
                \item [s]$\rightarrow 011$
                \item [r]$\rightarrow 110$
                \item [t]$\rightarrow 110$
                \item [-]$\rightarrow 100$
            \end{itemize}
        \end{column}

    \end{columns}
\end{frame}

\begin{frame}[t]
    \frametitle{Análise do armazenamento}
    \begin{center}
        \begin{tabular}{clccc}
            char               & bin     & bits & freq & total \\
            \hline
            \pause U           & 10000   & 5    & 1    & 5     \\
            \pause d           & 1000010 & 7    & 1    & 7     \\
            \pause m           & 1000011 & 7    & 1    & 7     \\
            \pause p           & 0010    & 4    & 2    & 8     \\
            \pause o           & 0011    & 4    & 2    & 8     \\
            \pause g           & 0000    & 4    & 2    & 8     \\
            \pause a           & 0001    & 4    & 3    & 12    \\
            \pause i           & 1001    & 4    & 3    & 12    \\
            \pause e           & 100010  & 6    & 4    & 24    \\
            \pause s           & 011     & 3    & 4    & 12    \\
            \pause r           & 100     & 3    & 6    & 18    \\
            \pause t           & 111     & 3    & 6    & 18    \\
            \pause -           & 100     & 3    & 7    & 21    \\
            \hline
            \pause Comprimido  &         &      &      & 160   \\
            \pause Comp. médio &         &      &      & 3.81  \\
        \end{tabular}
    \end{center}

\end{frame}

\begin{frame}[t]
    \frametitle{Análise do armazenamento}
    Antes do armazenamento:

    $$S = 42\times 8 = 336\ \text{bits}$$

    \pause Depois do armazenamento:

    $$s=5+7+7+8+8+8+12+12+24+12+18+21 = 160\ \text{bits}$$

    \pause Tabela de Símbolos:

    $$t = 13\times 8 = 104\ \text{bits}$$

    \pause Compactado:

    $$c = S-(s+t) = 336-(160+104) = 72\ \text{bits}$$

    \pause Taxa de compressão:

    $$Tx = \frac{c}{T} = \frac{72}{336}=21.42\%$$


\end{frame}
\input binhex
\begin{frame}[t]
    \frametitle{Resultados}
    % Antes da compressão(em decimal):\vfill

    % 85-109-95-112-114-97-116-111-95-100-101-95-116-114-105-103-111-95-112-97-114-97-95-116-114-101-115-95-116-105-103-114-101-115-95-116-114-105-115-116-101-115\vfill

    Antes da compressão(em binário):\vfill

    \nbinary{8}{85}\nbinary{8}{109}\nbinary{8}{95}\nbinary{8}{112}\nbinary{8}{114}\nbinary{8}{97}\nbinary{8}{116}\nbinary{8}{111}

    \nbinary{8}{95}\nbinary{8}{100}\nbinary{8}{101}\nbinary{8}{95}\nbinary{8}{116}\nbinary{8}{114}\nbinary{8}{105}\nbinary{8}{103}

    \nbinary{8}{111}\nbinary{8}{95}\nbinary{8}{112}\nbinary{8}{97}\nbinary{8}{114}\nbinary{8}{97}\nbinary{8}{95}\nbinary{8}{116}

    \nbinary{8}{114}\nbinary{8}{101}\nbinary{8}{115}\nbinary{8}{95}\nbinary{8}{116}\nbinary{8}{105}\nbinary{8}{103}\nbinary{8}{114}

    \nbinary{8}{101}\nbinary{8}{115}\nbinary{8}{95}\nbinary{8}{116}\nbinary{8}{114}\nbinary{8}{105}\nbinary{8}{115}\nbinary{8}{116}

    \nbinary{8}{101}\nbinary{8}{115}\vfill

    Após a compressão:\vfill

    1000010000111000010110000111000111001000010010100110110100100000
    0111000010000111000011001101100100111001101001000011001001110011
    0110100101111001001101010101011011010101111101110000011100100110
    00010111010001101111

\end{frame}


\begin{frame}[t]
    \frametitle{Exercícios}
    \begin{enumerate}[label=\textcircled{\scriptsize\arabic*}]
        \item Calcule a taxa de compressão de: "ABABABACBABABABA".
        \item Utilize o resultado da compressão dos dados para representar: "ACCCCCCCCC".
        \item Dado um texto qualquer, a compressão é única? Explique o porquê.
        \item Construa a árvore de Huffman a partir da frequência relativa das letras da lingua portuguesa.

              \begin{table}[htbp]
                  \centering

                  \begin{tabular}{lc|lc|lc|lc|lc}
                      \toprule
                      \textbf{\$} & \textbf{f (\%)} & \textbf{\$} & \textbf{f (\%)} & \textbf{\$} & \textbf{f (\%)} & \textbf{\$} & \textbf{f (\%)} & \textbf{\$} & \textbf{f (\%)} \\
                      \midrule
                      A           & 14.31           & N           & 5.86            & L           & 2.78            & F           & 0.97            & W           & 0.01            \\
                      E           & 12.46           & M           & 4.99            & P           & 2.52            & H           & 0.74            & Y           & 0.01            \\
                      O           & 10.73           & U           & 4.86            & V           & 1.52            & Z           & 0.55            &             &                 \\
                      I           & 8.99            & T           & 4.36            & G           & 1.25            & J           & 0.53            &             &                 \\
                      S           & 7.07            & D           & 3.53            & Q           & 1.20            & X           & 0.48            &             &                 \\
                      R           & 6.54            & C           & 3.13            & B           & 1.04            & K           & 0.02            &             &                 \\
                      \bottomrule
                  \end{tabular}
              \end{table}
    \end{enumerate}
\end{frame}

\begin{frame}[t]
    \frametitle{Exercícios}
    \begin{enumerate}[label=\textcircled{\scriptsize\arabic*}]
        \setcounter{enumi}{4}
        \item De acordo com a arvore construída na questão anterior, determine a codificação de: `ABABABACBABABABA'. Apresente a tabela de símbolos.
        \item Uma palavra foi codificada usando o código de Huffman, tendo-se obtido a sequência binária:\ $1 0 1 1 1 0 1 1 0 1 0 1 1 1 0 0 1 1 1 0 1 0 0$

              \begin{table}[htbp]
                  \centering
                  \begin{tabular}{c|c}
                      \textbf{Caractere} & \textbf{Probabilidade} \\
                      \hline
                      $P(A)$             & 0,26                   \\
                      $P(B)$             & 0,09                   \\
                      $P(C)$             & 0,08                   \\
                      $P(D)$             & 0,01                   \\
                      $P(E)$             & 0,07                   \\
                      $P(I)$             & 0,22                   \\
                      $P(L)$             & 0,01                   \\
                      $P(R)$             & 0,23                   \\
                      $P(T)$             & 0,03                   \\
                  \end{tabular}
              \end{table}
              Sabe-se que o $I$ é codificado como sendo 00. Qual é a palavra?
    \end{enumerate}
\end{frame}
\end{document}

