\documentclass[12pt]{beamer}
\usetheme{Frankfurt}
\usecolortheme{whale}
\usefonttheme{structurebold}
\usepackage{tcolorbox} % Pacote para criar caixas personalizadas
\usepackage{listings}
\usepackage{xcolor}

\newcommand{\CC}{C\nolinebreak\hspace{-.05em}\raisebox{.4ex}{\tiny\bf +}\nolinebreak\hspace{-.10em}\raisebox{.4ex}{\tiny\bf +}}
\def\CC{{C\nolinebreak[4]\hspace{-.05em}\raisebox{.4ex}{\tiny\bf ++}}}

% Configuração do ambiente lstlistings para o Bash
% Configuração do ambiente lstlistings para o Bash
\lstset{
  language=bash,              % Linguagem do código
  basicstyle=\ttfamily,       % Estilo básico da fonte
  backgroundcolor=\color{gray!10}, % Cor de fundo
  frame=none,                 % Nenhuma moldura ao redor do código
  numbers=none,               % Sem números de linha
}

\definecolor{caixa}{RGB}{0, 200, 200} % Definindo a cor ciano
\definecolor{codigo}{RGB}{0, 0, 0} % Definindo a cor ciano
\definecolor{opt}{RGB}{100, 100, 100} % Definindo a cor ciano
\newcommand{\codigoEDescricao}[2]{%
  \begin{tcolorbox}[colback=gray!10,colframe=caixa,title=\textbf{\textcolor{codigo}{#1}}] % 
  \par
  #2
  \end{tcolorbox}%
}

\title{GDB}
\subtitle{GNU Debugger}
\author{Prof. Kennedy Reurison Lopes}
\date{\today}

\begin{document}

\frame{\titlepage}

\begin{frame}[t]
    \frametitle{Introdução}
    O \textbf{GDB} (GNU Project Debugger) é uma ferramenta para:\vfill
    \begin{itemize}
        \item Observar um programa enquanto este executa
        \item Ver o estado no momento que a execução falha
    \end{itemize}
    \vfill Permite:
    \begin{itemize}
        \item Iniciar a execução de um programa
        \item Executar linha-a-linha
        \item Especificar pontos de paragem
        \item Imprimir valores de variáveis
    \end{itemize}\vfill
    Suporta C, C++, Objective-C, Ada e Pascal (entre outras linguagens)
\end{frame}

\begin{frame}[fragile]
    \frametitle{Compilação para depuração}

    Considere que você possui um arquivo com o código fonte escrito em C com o nome \emph{main.c}.\vfill

    \textbf{Compilação Normal:}

    \begin{lstlisting}
    $ gcc -o main main.c
    \end{lstlisting}


    \textbf{Compilação em modo de depuração:}

    \begin{lstlisting}
    $ gcc -o main main.c -g
    \end{lstlisting}

\end{frame}

\newcommand{\mm}{\texttt{--}}
\begin{frame}[fragile,t]
    \frametitle{Execução\footnote{Cinza$\rightarrow$ opcional}}

    \codigoEDescricao{\$ gdb main \textcolor{opt}{core dump}}
    {Comando para executar um arquivo depurável}

    \codigoEDescricao{\$ gdb \mm args \underline{program} \underline{args\ldots}}
    {Execução de um arquivo depurável com parâmetros.}

    \codigoEDescricao{\$ gdb \mm pid \underline{pid}}{Executa em modo depuração um programa que já está em execução}
\end{frame}

\begin{frame}
    \frametitle{Execução}
    \codigoEDescricao{set args \underline{args\ldots}}
    {Modifica os argumentos a serem passados para o programa}

    \codigoEDescricao{run}
    {Roda o programa a ser depurado}

    \codigoEDescricao{kill}
    {Finaliza o programa em depuração}

\end{frame}


\begin{frame}
    \frametitle{Breakpoints}
    \codigoEDescricao{break \underline{where}}
    {Cria um novo breakpoint}

    \codigoEDescricao{delete \underline{breakpoint\#}}
    {Remove um breakpoint}

    \codigoEDescricao{enable \underline{breakpoint\#}}
    {Habilita um breakpoint desabilitado}

    \codigoEDescricao{Clear}
    {Apaga todos os breakpoints}

\end{frame}


\begin{frame}[t]
    \frametitle{Whatchpoints}
    \codigoEDescricao{whatch \underline{where}}
    {Cria um novo whatchpoint}

    \codigoEDescricao{delete/enable/disable \underline{breakpoint\#}}
    {Semelhante ao breakpoint}

\end{frame}

\begin{frame}[t]
    \frametitle{Condicionais}
    \codigoEDescricao{break/whatch \underline{where} if \underline{condicao}}
    {break/whatch caso determinada condição seja satisfeita}

    \codigoEDescricao{condition \underline{breakpoint\#}}
    {Modifica determinada condição de um break/whatch existente}

\end{frame}


\begin{frame}[t]
    \frametitle{Examinando a pilha}
    \codigoEDescricao{backtrace/where}
    {Exibe o rastreamento da pilha de execução}

    \codigoEDescricao{backtrace/where full}
    {Exibe o rastreamento da pilha de execução com informações sobre as variáveis locais}

    \codigoEDescricao{frame \underline{frame\#}}
    {Seleciona um determinado frame(um passo na pilha de execução) para operar sobre}

\end{frame}

\begin{frame}[t]
    \frametitle{Caminhando}
    \codigoEDescricao{step}
    {Caminha para a próxima instrução (linha de código), entrando nas funções}

    \codigoEDescricao{next}
    {Caminha para a próxima instrução (linha de código), \textbf{não} entrando nas funções}

\end{frame}

\begin{frame}[t]
    \frametitle{Caminhando}

    \codigoEDescricao{finish}
    {Continua até que a função corrente retorne}

    \codigoEDescricao{continue}
    {Continua a execução normal}

\end{frame}

\begin{frame}[t]
    \frametitle{Variáveis e Memória}
    \framesubtitle{Comando: print format var}
    \codigoEDescricao{Letras que podem ser usadas no format}
    {\begin{itemize}
            \item o (octal)
            \item x (hexadecimal)
            \item d (decimal)
            \item t (binário)
            \item f (float)
            \item a (endereço)
            \item i (instrução)
            \item c (caractere de 1 byte)
            \item s (string)
    \end{itemize}}
\end{frame}

\begin{frame}[t]
    \frametitle{Variáveis e Memória}
    \codigoEDescricao{display/format \underline{what}}{Mesmo que o print, mas imprime na tela a cada passo(step) de execução}
    \codigoEDescricao{undisplay \underline{display\#}}{Remove um display}
    \codigoEDescricao{enable/disable display \underline{display\#}}{Remove um display}
\end{frame}

\begin{frame}[t]
    \frametitle{Código Fonte}
    \codigoEDescricao{list,list \textcolor{opt}{filename:function}, list \textcolor{opt}{filename:line\_number}}{Mostra o código fonte que está sendo depurado}

\end{frame}

\begin{frame}[t]
    \frametitle{Informações}
    \codigoEDescricao{info args}{imprime os argumentos para a função da corrente pilha de execução}
    \codigoEDescricao{info breakpoints}{imprime os breakpoints e whatchpoints}
    \codigoEDescricao{info display}{imprime os breakpoints e whatchpoints}
\end{frame}

\begin{frame}[t]
    \frametitle{Informações}
    \codigoEDescricao{info local}{imprime as variáveis locais do frame selecionado}
    \codigoEDescricao{whatis \underline{variable name}}{Imprime o tipo da variável}
\end{frame}


\end{document}