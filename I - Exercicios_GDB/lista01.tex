\documentclass[12pt]{article}
\usepackage[left=1.5cm,right=1.5cm,top=2cm,bottom=2cm]{geometry} % Reduzindo as margens

\usepackage{amsmath}
\usepackage{tikz}
\usepackage{enumitem}

\usepackage[brazilian]{babel}
\usepackage{datetime2}

\usepackage{listings} % Pacote para exibir código fonte

\lstdefinestyle{codigoC}{
  language=C,
  basicstyle=\ttfamily\small,
  numbers=left,
  numberstyle=\tiny,
  numbersep=5pt,
  keywordstyle=\color{blue},
  morekeywords={int, char, float, double, if, else, for, while, return},
  stringstyle=\color{red!80!black!80},
  showstringspaces=false,
  breaklines=true,
  frame=single
}

\lstset{language=C++,
                basicstyle=\ttfamily,
                keywordstyle=\color{blue}\ttfamily,
                stringstyle=\color{red}\ttfamily,
                commentstyle=\color{green!60!black},
                morecomment=[l][\color{black!50}]{\#}
}

\usepackage{fancyhdr}
\pagestyle{fancy}

\pgfmathsetseed{\number\pdfrandomseed}

\newcommand{\randominteger}[2]{\pgfmathrandominteger{\random}{#1}{#2}\random}

\newcommand{\sequencia}[2]{
\foreach \i in {1,...,#1} {
        \randominteger{10}{#2}
    }
}
% Cabeçalho
\newcommand{\professor}{Kennedy Lopes\\Marcos Mikael}
\newcommand{\turma}{Laboratório de Algoritmos\\ e Estrutura de Dados II}
\newcommand{\ano}{2023\\}
\newcommand{\periodo}{3º Semestre}

% Configuração do cabeçalho
\fancyhead{}
\fancyhead[L]{\professor}
\fancyhead[C]{\turma}
\fancyhead[R]{\ano \periodo}

\setlength{\headheight}{14.49998pt}
\addtolength{\topmargin}{-2.49998pt}

\begin{document}
\begin{center}
    \LARGE \textbf{Lista de exercícios (GDB)}
\end{center}

\subsection*{Instalação e Confuguração:}
\begin{itemize}
    \item Instale o GCC em seu sistema operacional;
    \item Instale o GDB em seu sistema operacional;
    \item Compile um código simples em linguagem C com depuração (usando a flag -g);
    \item Coloque um breakpoint em uma determinada linha de código e execute o programa até esse ponto.
    \item Execute o programa no GDB usando o comando \emph{run};
\end{itemize}

\subsubsection*{Código 1 - Fibonacci}
\begin{lstlisting}[language=C++]
    int fibonacci(int n) {
        if (n <= 1) {
            return n;
        }
        else {
            return fibonacci(n - 1) + fibonacci(n - 2);
        }
    }
        \end{lstlisting}
\begin{enumerate}[label=\alph*)]
    \item Depure o código para descobrir quantas vezes o \textbf{fibonacci(5)} é calculado (requisitado).
    \item Descubra através da depuração, qual a maior quantidade de níveis (\emph{frames}) esse código alcança.
    \item Depure o código até o momento que o primeiro fibonacci(14) é calculado e o executável irá começar a calcular o fibonacci(13).
\end{enumerate}

\newpage
\subsubsection*{Código 2 - Operação Matemática}
\begin{lstlisting}[style=codigoC]
    #include <stdio.h>

    // Funcao para calcular algo
    int operacaoMatematica(int a, int b) {
        int r;
    
        while (b != 0) {
            r = a % b;
            a = b;
            b = r;
        }
    
        return a;
    }
    
    int main() {
        int num1, num2;
        
        printf("Digite o primeiro numero: ");
        scanf("%d", &num1);
        
        printf("Digite o segundo numero: ");
        scanf("%d", &num2);
    
        int res = operacaoMatematica(num1, num2);
    
        printf("A operacao de %d e %d e: %d\n", num1, num2, res);
    
        return 0;
    }
    
\end{lstlisting}
\begin{enumerate}[label=\alph*)]
    \item Execute o algoritmo em modo de depuração.
    \item Descubra através da depuração, qual a maior profundidade de (\emph{frames}) esse código alcança.
    \item Descubra o que ocorre quando um parâmetro é $0$.
    \item Descubra o que ocorre quando os dois parâmetros apresentados são iguais e maiores que $1$.
\end{enumerate}

\newpage
\subsubsection*{Código 3 - Números primos}
\begin{lstlisting}[style=codigoC]
    #include <stdio.h>

    // Funcao recursiva para verificar se um numero e primo
    int ehPrimoRecursivo(int n, int i) {
        if (n <= 2) {
            return (n == 2);
        }
        
        if (n % i == 0) {
            return 0;
        }
        
        if (i * i > n) {
            return 1;
        }
        
        return ehPrimoRecursivo(n, i + 1);
    }
    
    // Funcao para imprimir numeros primos em um intervalo
    void imprimirPrimosIntervalo(int inicio, int fim) {
        if (inicio > fim) {
            return;
        }
        
        if (ehPrimoRecursivo(inicio, 2)) {
            printf("%d ", inicio);
        }
        
        imprimirPrimosIntervalo(inicio + 1, fim);
    }
    
    int main() {
        int inicio, fim;
    
        printf("Digite o inicio do intervalo: ");
        scanf("%d", &inicio);
    
        printf("Digite o fim do intervalo: ");
        scanf("%d", &fim);
    
        printf("Numeros primos entre %d e %d: ", inicio, fim);
        imprimirPrimosIntervalo(inicio, fim);
        printf("\n");
    
        return 0;
    }
        
\end{lstlisting}
\begin{enumerate}[label=\alph*)]
    \item Através da depuração, compreenda como o código funciona.
    \item Desenhe em formato de uma árvore de recursão os passos para o cálculo da procura dos números primos de $10$ a $20$ (usando a depuração).
    \item Conte a quantidade de chamadas recursivas para realizar o resultado de b).
    \item Depure o algoritmo com os valores de entrada de $1000$ e $10000$. Utilize o depurador para interromper a depuração quando o 240º número primo aparecer.
\end{enumerate}

\newpage

\subsubsection*{Código 4 - Função aleatória}
A figura a seguir representa um círculo inscrito em uma circunferência. A origem (0,0) do centro de coordenadas coincide com a origem do círculo e o centro do quadrado. Foi desenvolvido um algoritmo para gerar pontos aleatórios que aparecerão nos limites deste quadrado. O objetivo é selecionar apenas os pontos que estão dentro do círculo. A seguir, apresentamos o código e algumas questões relacionadas ao mesmo.

\begin{center}
    \begin{tikzpicture}
        \def\lados{7}
        \def\raio{\lados/2}
        \draw[fill=blue!20] (0,0) rectangle ++(\lados,\lados);
        \draw[fill=red!60] (\raio,\raio) circle (\raio);

        \draw (\lados/2,\lados/2) -- (\lados,\lados/2);
        \node at (1.5*\raio,\raio + 0.3) {R = 50};
        \draw[|-|] (0,-1) -- (\lados,-1);
        \node at (\raio, -1.3) {L = 100};
        \node at (0, -0.3) {\textbf{(-50,-50)}};
        \node at (\lados, \lados + 0.3) {\textbf{(50, 50)}};
        \node at (\lados, -0.3) {\textbf{(50,-50)}};
        \node at (0, \lados+0.3) {\textbf{(-50,50)}};
        \node at (\raio,\raio) {\textbf{$\times$}};
    \end{tikzpicture}
\end{center}

\begin{lstlisting}[language=C++]
    int funcRecursiva(int n) {
        if (n == 0) {
            return 1;
        }
        return funcRecursiva(n-1) + 1/funcRecursiva(n-1);
    }
\end{lstlisting}
\begin{enumerate}[label=\alph*)]
    \item Através da depuração, realize a interrupção do algoritmo toda vez que o número aleatório é maior que $0$.
    \item Utilize a depuração para realizar interrupções no algoritmo somente quando os pontos estiver na região externa ao círculo.
    \item Utilize a depuração para realizar interrupções no algoritmo somente quando os pontos estiver na região interna ao círculo.
\end{enumerate}


\end{document}
